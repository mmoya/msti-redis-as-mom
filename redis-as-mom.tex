\documentclass[spanish,xcolor=dvipsnames]{beamer}
\usetheme{Frankfurt}

\definecolor{links}{HTML}{2A1B81}
\hypersetup{colorlinks,linkcolor=,urlcolor=links}

\usepackage{hyperref}
\usepackage[utf8x]{inputenc}
\usepackage{makeidx}
\usepackage{color}
\makeindex

\hypersetup{
  pdftitle={Redis como Middleware Orientado a Mensajes (MoM)},
  pdfauthor={Maykel Moya},
  pdfproducer=PDFLaTeX,
}

\include{pygments}

\beamertemplatenavigationsymbolsempty

\title{Redis como \textit{Middleware} Orientado a Mensajes (MoM)}
\author[Maykel Moya]{Maykel Moya}
\institute[URJC]{
  Máster en Sistemas Telemáticos e Informáticos \\
  Universidad Rey Juan Carlos \\
  Madrid
}
\date{abril 2012}

\begin{document}

\begin{frame}[plain]
  \titlepage
\end{frame}

\begin{frame}[plain]
  \vspace{6cm}
  \tiny
  \hfill \copyright 2012 Maykel Moya \\
  \hfill Algunos derechos reservados \\
  \hfill Este trabajo se distribuye bajo la licencia \\
  \hfill \includegraphics[width=15mm]{images/by-nc-sa.png} \\
  \hfill Creative Commons Attribución-NoComercial-CompartirIgual \\
  \hfill disponible en \url{http://creativecommons.org/licenses/by-nc-sa/3.0/deed.es} \\
\end{frame}

\begin{frame}[plain]{Agenda}
  \tableofcontents[]
\end{frame}

\section{Memcached}

\begin{frame}{En el principio fue Memcached}
  \begin{itemize}\addtolength{\itemsep}{0.5\baselineskip}
    \item Servidor de caché (acceso por la red)
    \item Sólo comandos \texttt{GET} y \texttt{SET} (simple)
    \item Alto rendimiento
    \item Almacenamiento en memoria (volátil)
    \item Almacenamiento basado en llave/valor
    \item Software Libre
    \item \url{http://memcached.org/}
  \end{itemize}
\end{frame}

\begin{frame}{Uso típico}
  Sólo comandos \texttt{GET} y \texttt{SET}
  \begin{columns}
    \begin{column}{0.50\textwidth}
      \includegraphics[width=\linewidth]{images/memcache-interactive-session.png}
    \end{column}
    \pause
    \begin{column}{0.40\textwidth}
      \tiny
      \include{memcache-example}
    \end{column}
  \end{columns}
\end{frame}

\section{Redis}

\begin{frame}{Luego Redis}
  \begin{itemize}\addtolength{\itemsep}{0.5\baselineskip}
    \item Servidor de caché (acceso por la red)
    \item Alto rendimiento
    \item Almacenamiento persistente
    \item Almacenamiento basado en llave/valor
    \begin{itemize}
      \item Pero los valores pueden ser \textit{strings},
        \textit{hashes}, \textit{lists} entre otros
      \item ... por lo que se conoce también como Servidor de
        Estructuras de Datos
    \end{itemize}
    \item Software Libre
    \item \url{http://redis.io/}
  \end{itemize}
\end{frame}

\begin{frame}{Comandos}
  \begin{itemize}
    \item Además de los básicos \texttt{GET} y \texttt{SET}
  \end{itemize}
  \begin{itemize}
    \item \texttt{LPUSH}: Inserción de un elemento al principio de una lista (\textit{Left PUSH})
    \item \texttt{RPOP}: Eliminación de un elemento del final de una lista (\textit{Right POP})
    \item \texttt{BRPOP}: Idem pero bloqueante (\textit{Blocking Right POP})
    \item \texttt{PUBLISH}: Publicación en un canal
    \item \texttt{SUBSCRIBE}: Subscripción a un canal
    \item \texttt{UNSUBSCRIBE}: Cancelar subscripción a un canal
    \item \textit{muchos otros}
  \end{itemize}
  \begin{itemize}
    \item \url{http://redis.io/commands}
  \end{itemize}
\end{frame}

\section{Redis como MoM}

\begin{frame}{\textit{Middleware} Orientado a Mensajes (MoM)}
% https://en.wikipedia.org/wiki/Message-oriented_middleware
  \begin{block}{De Wikipedia:}
    \begin{quotation}
       ... infraestructura de software o hardware que permite el envío
       y recepción de mensajes entre componentes de un sistema
       distribuido
     \end{quotation}
  \end{block}
\end{frame}

\begin{frame}{Paradigmas de distribución en un MoM}
  \begin{center}
  \begin{tabular}{ l || p{2cm} | p{2cm} | p{2cm} }
                 & \scriptsize cuántos suscriptores reciben el mensaje?
                 & \scriptsize se retiene el mensaje?
                 & \scriptsize se puede implementar con Redis? \\
    \hline\hline
    \textit{queue}                & uno   & sí & sí \\
    \textit{topic}                & todos & no & sí \\
    \textit{durable subscription} & todos & sí & sí \\
  \end{tabular}
  \end{center}

  \pause

  \vspace{\baselineskip}

  ¿Cómo se implementa con Redis?

  \begin{itemize}\addtolength{\itemsep}{0.5\baselineskip}
    \item \textit{Queue}: Con \texttt{LPUSH} y \texttt{BRPOP}
    \pause
    \item \textit{Topic}: Con \texttt{SUBSCRIBE} y \texttt{PUBLISH}
    \pause
    \item \textit{Durable Subscription}: Con \texttt{SUBSCRIBE}, \texttt{PUBLISH}
      y \textit{Sorted Sets}, ver \url{http://j.mp/JBjwdT}.
  \end{itemize}
\end{frame}

\section{Demo}

\begin{frame}
  \begin{center}
  \Huge
  Y ahora el demo!
  \end{center}
\end{frame}

\begin{frame}[plain]{Preguntas}
   \includegraphics[width=\linewidth]{images/question.png}
\end{frame}

\begin{frame}[plain]{Referencias}
  \begin{enumerate}\addtolength{\itemsep}{0.5\baselineskip}
    \item \href{http://www.darkcoding.net/software/choosing-a-message-queue-for-python-on-ubuntu-on-a-vps/}
               {Choosing a message queue for Python on Ubuntu on a VPS}.
    \item \href{http://stackoverflow.com/questions/7871526/is-non-blocking-redis-pubsub-possible}
               {Is non-blocking Redis pubsub possible?}.
    \item \href{http://redis.io/topics/pubsub}
               {Redis Pub/Sub}.
    \item \href{http://stackoverflow.com/a/7672670/253049}
               {Does the redis pub/sub model require persistent connections to redis?}.
    \item \href{http://pkghosh.wordpress.com/2012/03/28/redis-as-messaging-middleware/}
               {Redis as Messaging Middleware}.
    \item \href{http://blog.joshsoftware.com/2011/01/03/do-you-need-a-push-notification-manager-redis-pubsub-to-the-rescue/}
               {Do you need a Push Notification Manager? – Redis PubSub to the rescue}.
  \end{enumerate}
\end{frame}

\end{document}
